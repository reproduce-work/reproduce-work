\documentclass[12pt]{article}\usepackage[english]{babel}\usepackage{xcolor}\usepackage[hmargin=1in,vmargin=1in]{geometry}\usepackage{amsmath}\usepackage{unicode-math}\usepackage[round,sort,comma]{natbib}\bibliographystyle{apa}\usepackage{setspace}\usepackage{graphicx}\usepackage{caption}\usepackage{subcaption}\usepackage[colorlinks=true, allcolors=blue]{hyperref}\usepackage{float}\usepackage{booktabs}\usepackage{titlesec}\usepackage{fontspec}\newcommand{\addperiod}[1]{#1.$\;$}\titlespacing{\section}{0pt}{\parskip}{-\parskip}\titleformat{\subsection}[runin]{\normalsize\bfseries}{\thesubsection}{1em}{\addperiod}\titleformat{\subsubsection}[runin]{\normalfont\normalsize\itshape}{\thesubsubsection}{1em}{\addperiod}\titlespacing{\subsubsection}{14pt plus 4pt minus 2pt}{0pt}{0pt plus 2pt minus 2pt}\setlength{\parindent}{1em}\makeatletter\g@addto@macro \normalsize {\setlength\abovedisplayskip{3pt plus 5pt minus 2pt}\setlength\belowdisplayskip{3pt plus 5pt minus 2pt}}\makeatother\newcommand{\comment}[1]{}\begin{document}\pagenumbering{gobble}\begin{center}{\fontsize{16}{16}\selectfont\bfseries \texttt{reproduce.work}: A framework to facilitate cross-platform computational reproducibility in scientific publishing}\\\vspace{5mm}\begin{table}[!ht]\begin{center}\begin{tabular}{c c }\shortstack{ Alex P. Miller \\USC Marshall School of Business \\alex.miller@marshall.usc.edu }\end{tabular}\end{center}\end{table}\vspace{5mm}\emph{Last updated: \today}\vspace{4mm}\hyphenpenalty=10000 \exhyphenpenalty=10000\abstract{In metascience, computational reproduction is the process of reproducing the results of a scientific paper using the data and code provided by the authors of the paper. This subject sits within the broader context of "reproducibility" in scientific  research, which has been core to the philosophy of science for decades. However, the practice of science has fallen woefully short of meeting even basic standards toward true and widespread reproducibility. In this project, we focus primarily on addressing the narrow problem of computational reproucibility. We propose a framework for facilitating computational reproducibility in scientific publishing, which we call reproduce.work. The standards of verification are designed to be cross-platform, and to work with any programming language and nearly any existing workflow. We highlight the distinction between open and reproducibile practices and showcase how our software framework encourages both simultaneously. This entire paper can be reproduced on any machine that can execute a containerized image using our protocol. We conclude by discussing the potential of the framework for improving rigor and fidelity of computational science for both producers and consumers of published work.
}\\\vspace{2cm}{\scriptsize \noindent Notes: reproduce.work/v0.0.1  \raisebox{-1mm}{\includegraphics[width=5mm]{../../nbs/img/logo.png}}}\hyphenpenalty=50 \exhyphenpenalty=50\vspace{10mm}\end{center}\newpage\doublespacing\pagenumbering{arabic}\setcounter{page}{1}%%@@LOWDOWN_CONTENT@@%%\bibliography{latex/bibliography}\end{document}